% E. Dunham -- Resume
% Contents Copyright (C) 2014 - 2016, E. Dunham

% LaTeX code for rendering the resume is distributed under the MIT license.
% See LICENSE.txt. It means you can use the code for whatever you want,
% including your own resume, but I'm not liable if it catches your computer on
% fire.

% Template originally developed by E. Dunham
% https://github.com/edunham/resume/blob/master/resume.tex

\usepackage[normalem]{ulem} % For the underlines
\usepackage[compact]{titlesec} % Shrink default spacings
\usepackage{tabto} % For aligning skills section
\usepackage{parskip}
\textwidth=7in
\textheight=10.5in
\topmargin -1in % Reclaim the default whitespace from top of page
\oddsidemargin -.25in % Reclaim whitespace on left, make it look centered
\pagenumbering{gobble} % Don't number pages
\setlength{\parindent}{0pt} % Don't indent paragraphs
\setlength{\parskip}{5pt}
\newcommand{\heading}[1]{
%    \section*{\centering\uline{\hfill #1 \hfill }} % Center the headings
    \section*{\uline{\hfill #1 }} % Right-align the headings
}
\newcommand{\squish}{
    \setlength{\itemsep}{0.5pt}
    \setlength{\parskip}{0pt} % tweak parskip value to adjust total height
    \setlength{\parsep}{0.5pt}
}
\newcommand{\when}[1]{ % naming this 'date' would conflict with builtins
    \hfill \texttt{#1}
}
\newcommand{\experience}[3]{ % place, optional title, date
    \ifx&#2&
        \item[{#1}]
    \else
        \item[{#1}, \emph{#2}]
    \fi
    \when{#3}
}
\newcommand{\contact}[3]{
  \hfill \emph{#1}

  \hfill \emph{#2}

  \hfill \emph{#3}
}
\newcommand{\addressee}[2]{
  \today

  {#1}

  {#2}
}
\newcommand{\reqs}[2]{
  {#1} & {#2}\\
}
\newcommand{\skill}[2]{
    \textbf{#1} \tabto{2.5in} #2
}
% Write C++ all fancy-like
% http://www.parashift.com/c++-faq-lite/latex-macros.html
\newcommand{\CPP}{
    C\hspace{-.05em}\raisebox{.4ex}{\tiny\bf +}\hspace{-.10em}\raisebox{.4ex}{\tiny\bf +}
}
